%\documentclass[preprint]{aastex}
\documentclass[twolcolumn,apj]{emulateapj}
\usepackage{ctable}
\usepackage{amsmath}
\usepackage{graphicx}
%\usepackage[figuresright]{rotating}
%\usepackage{rotating}
%\usepackage{natbib}
%\usepackage{pdflscape}
%\usepackage{lscape}
%\citestyle{aa}

\def\b{\mathbf{b}}
\def\k{\mathbf{k}}
\def\r{\mathbf{r}}
\def\q{\mathbf{q}}
\def\b{\mathbf{b}}
\def\kp{\mathbf{k}^\prime}
\def\kpp{\mathbf{k}^{\prime\prime}}
\def\V{\mathbb{V}}
\def\At{\tilde{A}}
\def\Vt{\tilde{V}}
\def\Tt{\tilde{T}}
\def\tb{\langle T_b\rangle}
\newcommand{\vis}{\mathbf{v}}
\newcommand{\x}{\mathbf{x}}
\newcommand{\xhat}{\hat{\mathbf{x}}}
\newcommand{\A}{\mathbf{A}}
\newcommand{\N}{\mathbf{N}}
\newcommand{\rhat}{\hat{\mathbf{r}}}
\newcommand{\acl}[1]{{\color{red} \textbf{[ACL:  #1]}}}
\newcommand{\mep}[1]{{\color{blue} \textbf{[MEP:  #1]}}}

\renewcommand{\topfraction}{0.85}
\renewcommand{\bottomfraction}{0.1}

\begin{document}

%\title{What Power Spectrum Measurements From the Next Generation of 21~cm Experiments Can Teach Us About the Epoch of Reionization}
\title{Global signal interferometer (XXX: replace with better title)}

\author{Morgan E. Presley\altaffilmark{1},
}
\altaffiltext{1}{Department of Astrophysical Sciences, Princeton University, Princeton NJ}

\begin{abstract}
Really great abstract
\acl{Hey, look at these fun commands.}
\mep{Here's yours.}
\end{abstract}


\keywords{reionization, dark ages, first stars --- techniques: interferometric}

\section{Introduction}
\begin{itemize}
\item Say why 21cm is important.
\item Narrow focus to global signal.
\item Survey of current instruments and efforts.
\item Challenge of foregrounds.
\item Why an interferometer might be helpful plus why we're not crazy (Ekers + Rots Theorem?).  Angular info helps with foreground suppression.
\item Preview results (exquisite extraction; great foreground mitigation; high significance detection).
\item ``The rest of this paper is organized as follows..."
\end{itemize}

\section{Why we're really not crazy}
\begin{itemize}
\item Cartoon / qualitative picture?
\item Simple eqn. showing that there is non-zero response?
\item Intuitive, qualitative description of what sort of arrays are good.  (Preview for later sections).
\item Discussion of noise bias and other problems with single dipole experiments?
\end{itemize}

\section{Mathematical formalism}
\begin{itemize}
\item How does one actually get from visibilities to a global signal measurement.
\item The error statistics that come for free.
\item Quantification of leakage from other spherical harmonics.
\end{itemize}

In order to obtain a global signal measurement from visibilities, it is necessary to invert the linear equation 
\begin{equation}
\mathbf{y} = \textrm{Q} \mathbf{x} + \mathbf{n}
\end{equation}
\acl{Let's make matrices bold as well.} where $\mathbf{y}$ is a vector of length $N_{\textrm{bl}}$ containing the visibilities measured at different baselines, $\mathbf{n}$ is the contribution of noise from both foregrounds and systematics, and $\mathbf{x}$ is a projection of the true sky onto the spherical harmonics, $Y_{lm}(\theta,\phi)$. As such, $\mathbf{x}$ has length $(l_{\textrm{max}}+1)^2$, where $l_{\textrm{max}}$ is the largest $l$ \acl{Latex actually has a special command for the script $\ell$ that's used for spherical harmonics: ``\textbackslash ell"} value used in the model of the true sky. The matrix $\textrm{Q}$ is the beam response of an array at different baselines (rows) and different spherical harmonics (columns). $\textrm{Q}$ is determined by 
\begin{equation}
\textrm{Q}_{j,lm} = \int d \boldsymbol \theta^2 A(\boldsymbol \theta) Y_{lm}(\boldsymbol \theta) e^{2\pi i \frac{\mathbf{b_\textit{j}}}{\lambda} \cdot \boldsymbol \theta}
\end{equation}
where $\boldsymbol \theta$ is the angular position on the sky, $A$ is the beam amplitude \mep{is that what this is called?} \acl{``Primary beam" is the term you're looking for.} for the antennas, $\mathbf{b_{\textit{j}}}$ is the $j$th baseline, and $\lambda$ is the wavelength of observation. Since $\textrm{Q}$ is not in general invertible, we cannot solve for $\mathbf{x}$ directly and instead can only recover an estimator for $\mathbf{x}$. We can define a general estimator of the form 
\begin{equation}
\mathbf{\hat x} = \textrm{M} \textrm{Q}^\dagger \textrm{N}^{-1} \mathbf{y}
\end{equation}
where N is the total noise covariance matrix and M is some matrix chosen by the data analyst. One useful choice for M is 
\begin{equation}
\textrm{M} = [\textrm{Q}^\dagger \textrm{N}^{-1} \textrm{Q}]^{-1}
\end{equation}
which is obtained through a least-squares derivation that minimizes $\chi^2 \equiv (\mathbf{y}-\textrm{Q} \mathbf{\hat x})^\dagger \textrm{N}^{-1} (\mathbf{y}-\textrm{Q} \mathbf{\hat x})$. Note that for this choice of M, the ensemble average of $\mathbf{\hat x}$ is $\mathbf{x}$.
%\begin{equation}
%\begin{split}
%<\mathbf{\hat x}> & = < [\textrm{Q}^\dagger \textrm{N}^{-1} \textrm{Q}]^{-1} \textrm{Q}^\dagger \textrm{N}^{-1} \mathbf{y} > \\
%& = [\textrm{Q}^\dagger \textrm{N}^{-1} \textrm{Q}]^{-1} \textrm{Q}^\dagger \textrm{N}^{-1} ( < \textrm{Q} \mathbf{x} > + <\mathbf{n} >) \\
%& =  [\textrm{Q}^\dagger \textrm{N}^{-1} \textrm{Q}]^{-1} [\textrm{Q}^\dagger \textrm{N}^{-1} \textrm{Q}] <\mathbf{x} > \\
%& = <\mathbf{x}>
%\end{split}
%\end{equation}
Using this fact, we can directly find the covariance of $\mathbf{\hat x}$ to be 
\begin{equation}
\textrm{Cov}(\mathbf{\hat x}) = [\textrm{Q}^\dagger \textrm{N}^{-1} \textrm{Q}]^{-1}
\end{equation}
\mep{Should I include the derivation? I'm guessing not since it wasn't difficult.} \acl{No need to derive it.  Just cite the Tegmark mapmaking paper.} [Also, not sure what you meant by quantification of leakage... so I'm skipping that for now.]

[talk about window function; also pseudo inverses]

[quantification of leakage -- show the green/pink plots of 1st row of window function bc 1st row is what determines the a00 we measure]

\acl{The tone of the writing is excellent, btw.}

\section{Exploration of arrays}
\begin{itemize}
\item Show plots of different arrays that we tried.
\item (Maybe spherical harmonic plots with fringe patterns overlaid on spherical harmonics).
\item Reasoning behind our crazier-looking arrays.
\item Rules of thumb for designing a global signal interferometer.
\end{itemize}

\section{Foregrounds and their mitigation}
\begin{itemize}
\item General characteristics of foregrounds.
\item Desire to avoid using too much spectral information because the signal itself is a spectrum.
\item How to use angular information.
\item (Modifications to array design to account for foregrounds).
\end{itemize}

The primary source of contamination in a global signal experiment comes from galactic foregrounds, which makes extraction of the global signal difficult for a number of reasons. Primarily, the galactic foregrounds are many orders of magnitude brighter than our expected signal, so finding the global signal through a simple spatial average over the sky would result in a measurement dominated by foregrounds. Also, since both the galactic foregrounds and the expected global re-ionization signal are spectrally smooth, foreground subtraction techniques will likely be unreliable. Therefore, we propose using angular information to differentiate galactic foregrounds from the global signal. 

\acl{An interesting point I just realized.  We can make the argument that since the normalization of $\mathbf{N}$ scales out of the expression for $\hat{\mathbf{x}}$, our foreground mitigation scheme is a very conservative one that only uses shape information, and not the detailed spectral form of our foreground model.}
\acl{Somewhere, we also need to address rotation synthesis}
\acl{Also need to add noise to simulations}.

\section{Monte carlo / foreground simulation results}
\begin{itemize}
\item $\mathbf{N}_\textrm{fg}$ might not be properly modeled, so we can't use $[\mathbf{Q}^\dagger \mathbf{N}^{-1} \mathbf{Q}]^{-1}$ as a reliable indicator of the errors.  (Although recall that it's a perfectly fine choice for extracting the signal).
\item To get reliable estimates...Monte Carlo!
\item Really great results! (Error bars and covariance on the recovered spectra).
\end{itemize}

\section{Fisher matrix results}
\begin{itemize}
\item Fisher matrix formalism for translating error statistics from recovered spectra to parameterizations of the signal.
\end{itemize}

\section{Conclusions}
\begin{itemize}
\item This is a great way to do this measurement.  It will produce lots of great science! And eternal glory!
\end{itemize}

\bibliographystyle{apj}
%\bibliography{nextgen}{}

\end{document}
