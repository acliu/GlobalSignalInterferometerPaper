%\documentclass[preprint]{aastex}
\documentclass[twolcolumn,apj]{emulateapj}
\usepackage{ctable}
\usepackage{amsmath}
\usepackage{graphicx}
%\usepackage[figuresright]{rotating}
%\usepackage{rotating}
%\usepackage{natbib}
%\usepackage{pdflscape}
%\usepackage{lscape}
%\citestyle{aa}

\def\b{\mathbf{b}}
\def\k{\mathbf{k}}
\def\r{\mathbf{r}}
\def\q{\mathbf{q}}
\def\b{\mathbf{b}}
\def\kp{\mathbf{k}^\prime}
\def\kpp{\mathbf{k}^{\prime\prime}}
\def\V{\mathbb{V}}
\def\At{\tilde{A}}
\def\Vt{\tilde{V}}
\def\Tt{\tilde{T}}
\def\tb{\langle T_b\rangle}
\newcommand{\vis}{\mathbf{v}}
\newcommand{\x}{\mathbf{x}}
\newcommand{\xhat}{\hat{\mathbf{x}}}
\newcommand{\A}{\mathbf{A}}
\newcommand{\N}{\mathbf{N}}
\newcommand{\rhat}{\hat{\mathbf{r}}}

\renewcommand{\topfraction}{0.85}
\renewcommand{\bottomfraction}{0.1}

\begin{document}

%\title{What Power Spectrum Measurements From the Next Generation of 21~cm Experiments Can Teach Us About the Epoch of Reionization}
\title{Global signal interferometer (XXX: replace with better title)}

\author{Morgan E. Presley\altaffilmark{1},
}
\altaffiltext{1}{Department of Astrophysical Sciences, Princeton University, Princeton NJ}

\begin{abstract}
Really great abstract
\end{abstract}


\keywords{reionization, dark ages, first stars --- techniques: interferometric}

\section{Introduction}
\begin{itemize}
\item Say why 21cm is important.
\item Narrow focus to global signal.
\item Survey of current instruments and efforts.
\item Challenge of foregrounds.
\item Why an interferometer might be helpful plus why we're not crazy (Ekers + Rots Theorem?).  Angular info helps with foreground suppression.
\item Preview results (exquisite extraction; great foreground mitigation; high significance detection).
\item ``The rest of this paper is organized as follows..."
\end{itemize}

\section{Why we're really not crazy}
\begin{itemize}
\item Cartoon / qualitative picture?
\item Simple eqn. showing that there is non-zero response?
\item Intuitive, qualitative description of what sort of arrays are good.  (Preview for later sections).
\item Discussion of noise bias and other problems with single dipole experiments?
\end{itemize}

\section{Mathematical formalism}
\begin{itemize}
\item How does one actually get from visibilities to a global signal measurement.
\item The error statistics that come for free.
\item Quantification of leakage from other spherical harmonics.
\end{itemize}

\section{Exploration of arrays}
\begin{itemize}
\item Show plots of different arrays that we tried.
\item (Maybe spherical harmonic plots with fringe patterns overlaid on spherical harmonics).
\item Reasoning behind our crazier-looking arrays.
\item Rules of thumb for designing a global signal interferometer.
\end{itemize}

\section{Foregrounds and their mitigation}
\begin{itemize}
\item General characteristics of foregrounds.
\item Desire to avoid using too much spectral information because the signal itself is a spectrum.
\item How to use angular information.
\item (Modifications to array design to account for foregrounds).
\end{itemize}

\section{Monte carlo / foreground simulation results}
\begin{itemize}
\item $\mathbf{N}_\textrm{fg}$ might not be properly modeled, so we can't use $[\mathbf{Q}^\dagger \mathbf{N}^{-1} \mathbf{Q}]^{-1}$ as a reliable indicator of the errors.  (Although recall that it's a perfectly fine choice for extracting the signal).
\item To get reliable estimates...Monte Carlo!
\item Really great results! (Error bars and covariance on the recovered spectra).
\end{itemize}

\section{Fisher matrix results}
\begin{itemize}
\item Fisher matrix formalism for translating error statistics from recovered spectra to parameterizations of the signal.
\end{itemize}

\section{Conclusions}
\begin{itemize}
\item This is a great way to do this measurement.  It will produce lots of great science! And eternal glory!
\end{itemize}

\bibliographystyle{apj}
%\bibliography{nextgen}{}

\end{document}
